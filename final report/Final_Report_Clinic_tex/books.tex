%%% Copyright (C) 2004 Claire M. Connelly and 
%%% the Department of Mathematics, Harvey Mudd College.
%%%
%%% This file is part of the sample thesis document provided to HMC
%%% mathematics students.
%%%
%%% See the COPYING document, which should accompany this
%%% distribution, for information about distribution and modification
%%% of the document and its components.

\chapter{Books}

\url{http://www.math.hmc.edu/computing/support/tex/} has some brief
reviews of a number of significant books about \tex and \latex.

My pick for the best introductory/reference book is the third edition
of George Gr\"{a}tzer's \emph{Math into \latex{}}
\citeyearpar{gratzer-mil}.\footnote{Which I edited.}  It's the only book I'm
aware of that discusses the latest version of AMS\latex in depth.  It
also has excellent reference tables and a thorough index.

Another book I highly recommend is Lyn Dupr\'{e}'s \emph{BUGS in
  Writing} \citeyearpar{dupre-bugs}.  Dupr\'{e} is one of Addison Wesley's
senior editors, and has edited many of the most significant books
published by Addison Wesley.  \emph{BUGS} is an accessible guide to
writing clearly and effectively.  It's the kind of book you leave in
the bathroom so you'll always have something interesting and amusing
to read.  Learning how to write better is almost a byproduct!

If you get serious about typesetting, and want to start doing some
fancy page design or want to be sure you're using the right kind of
type, Robert Bringhurst's \emph{The Elements of Typographic Style}
\citeyearpar{bringhurst-elements} will show you the way.


