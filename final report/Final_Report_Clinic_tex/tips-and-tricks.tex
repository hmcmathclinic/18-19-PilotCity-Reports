%%% Copyright (C) 2004 Claire M. Connelly and 
%%% the Department of Mathematics, Harvey Mudd College.
%%%
%%% This file is part of the sample thesis document provided to HMC
%%% mathematics students.
%%%
%%% See the COPYING document, which should accompany this
%%% distribution, for information about distribution and modification
%%% of the document and its components.

\chapter{Tips and Tricks}

\latex is a very complicated and powerful language.  As a result,
there are many sneaky aspects to it that will cause you problems if
you don't know about them.  This chapter covers some of these tricky
bits.


\section{Special Characters}%
\label{sec:special-chars}

\tex and \latex have a number of ``special'' characters that are
reserved for use by the language.  Using these characters in your
writing requires you to do a bit of extra work, as shown in
Table~\ref{tab:special-chars}.

\begin{table}
\begin{tabular}{llll}
\toprule
\multicolumn{2}{l}{Character}   & Function     & To Typeset\\
\midrule
\#          & octothorp      & Macro parameter character  & \verb+\#+\\
\$          & dollar sign    & Start/end inline math mode & \verb+\$+\\
\%          & percent sign   & Comment character          & \verb+\%+\\
\&          & ampersand      & Column separator           & \verb+\&+\\
\_          & underscore     & Subscripts, as in $x_2$    & \verb+\_+\\
\{, \}      & braces         & Parameters                 & \verb+\{, \}+\\
\~{}        & tilde          & Nonbreaking space          & \verb+\~{}+ \\
\^{}        & caret          & Superscripts, as in $x^2$  & \verb+\^{}+\\
\verb+\+    & backslash      & Starts commands   & \command{verb}\verb+|\|+ \\
\bottomrule
\end{tabular}
\caption[Special characters in \latex]{Special characters in \latex.}%
\label{tab:special-chars}
\end{table}


\section{Accents}

There are a variety of commands for producing diacritical accents, as in
\begin{quote}
  Paul Erd\H{o}s s'est reveill\'{e} t\^{o}t pour enseigner le
  fran\c{c}ais \`{a} son fr\`{e}re et sa s\oe{}ur.
\end{quote}

See Section~2.4.7 in \emph{Math into \LaTeX} \citep{gratzer-mil} for
information about accents (and some handy charts!).


\section{Comments and Spacing}

You can add comments to your source file that won't appear in your
typeset document by starting them with a \verb+%+.  Any line that
starts with a \verb+%+ will be ``commented out'', and won't be
interpreted.  You can also add a \verb+%+ at the end of a line, with
or without text, and it will make the end of the line (including the
linebreak!) disappear.

For example,
\begin{quote}
\begin{verbatim}
% This line is a comment line.
This line is not a comment line.

This line has a comment at the end%
% This line should be invisible.
of the line.
\end{verbatim}
\end{quote}
will typeset as
\begin{quote}
% This line is a comment line.
This line is not a comment line.

This line has a comment at the end%
% This line should be invisible.
of the line.
\end{quote}

Notice the lack of a space in ``endof'' on the last line of the
typeset output.  \tex expects to find  a carriage-return character at the end
of a line, and interprets that carriage return as an interword space.
If you comment out the end of a line, you also comment out the
carriage return on that line, and you'll have words run into one
another unless you have a space before the \verb+%+.

\tex collapses multiple spaces into one, and ignores whitespace at the
beginning of a line.  Thus
\begin{quote}
\begin{verbatim}
No spaces.

    Five spaces.

    A tab.
\end{verbatim}
\end{quote}
typesets as
\begin{quote}
No spaces.

    Five spaces.

    A tab.
\end{quote}

(The lines are indented because they are at the start of a paragraph.
You can suppress paragraph indentation with \command{noindent}.)

Paragraphs are delimited by two carriage returns (with or without
whitespace between them).


\section{Quotes and Dashes}

Because \tex was designed to do high-quality typesetting, it cares
about which quotation mark and dash you're using, and requires you to
specify the correct punctuation (although most text editors with
special \tex modes will do the substitution for you).

Open and close double quotes---`` and ''---are created by typing
\verb+``+ and \verb+''+, respectively.  The double-quote mark,
\verb+"+, is typeset as " (and is useful for abbreviating ``inches'',
as in 36").

Single-quotes, ` and ', are typed with \verb+`+ and \verb+'+.

There are three basic forms of dashes:
\begin{enumerate}
\item The hyphen, -, is typed as a single dash, \verb+-+
\item The en dash, --, is typed as two dashes, \verb+--+
\item The em dash, ---, is typed as three dashes, \verb+---+
\end{enumerate}

Hyphens are used in hyphenated words, as in ``high-quality''.  En
dashes are used to indicate ranges, as in ``there are 35--50 of
them''; balanced relationships, as in ``U.S.--European agreements'';
and for extended hyphenated phrases, such as ``reddish-green--colored
item''.   Em dashes are used to separate independent phrases, as in
``John believed---honestly believed---that he was right''; and to
indicate interruptions in dialogue, as in
\begin{quotation}
  ``I hear you---''

  ``No, you don't!''
\end{quotation}

Note that you shouldn't type spaces around any of these dashes---they
run directly against the words on either side, as in \verb+35--50+.


\section{Controlling Pagination}%
\label{sec:pagination}

Sometimes you may need to override \LaTeX's choices for line or page
breaks.  The \com{pagebreak} command causes \LaTeX\ to start a new
page immediately after the command appears.  The \verb|\\| command can
be used to tell \LaTeX\ where to break a line.

In general, you should let \LaTeX\ have its way, especially if your
document is going to be published by someone else, as they will
undoubtedly have many changes that will have to be made before your
document works for them.  If you do need to tinker with your
document's layout, you should avoid doing so until you're very nearly
done.  If you go back and add or remove text after forcing \LaTeX\ to
do your will, you may find that new blank spaces appear as a result of
your changes.

George Gr\"{a}tzer's \emph{Math into \LaTeX} \citeyearpar{gratzer-mil}
includes a chapter on preparing books that covers this topic in depth.


\section{Using Graphics with \pdftex}%
\label{sec:pdftex-graphics}

\pdftex supports PDF and JPEG as native graphic file formats.  EPS is
not directly supported---to use EPS figures with \pdftex, you must
first convert your EPS files to PDF.  

If you're using a graphics program such as Adobe Illustrator to
prepare your figures, just save them as PDF instead of (or in addition
to) EPS.

If you don't have access to the tool you used to create your images,
but you still need to convert them, you can use the program
\prog{epstopdf}.

\prog{epstopdf} writes to standard output by default, so you'll have
to redirect the output to a file, as in
\begin{quote}
\begin{verbatim}
unix% epstopdf foo.eps > foo.pdf
\end{verbatim}
\end{quote}

To convert a whole slew of files, you could use a command such as the
following (with the \prog{csh}):
\begin{quote}
\begin{verbatim}
unix% foreach f ( `find . -type f -name '*.eps'`)
foreach? eps2pdf $f -o=$f:r.pdf
foreach? end
\end{verbatim}
\end{quote}


\section{Fonts Look Fuzzy in PostScript or PDF Files}%
\label{sec:fuzzy-fonts}

When Donald~E. Knuth originally wrote \tex, most typesetting was done
by trained typesetters using expensive equipment to cast molten lead
into runs of type.  Knuth created his own font family, Computer
Modern, by writing a tool called \MF{}.  \MF{} reads in programs that
define various aspects of every character in a font, and generates
bitmap representations of those characters at a particular resolution,
ready for printing.

Unfortunately, bitmaps with resolutions suited for printing look
terrible on screen.  The solution is to use Type~1 PostScript fonts
instead of bitmaps. If you're using \pdftex (or \pdflatex), you get
Type~1 fonts without having to do anything special (but see
Section~\ref{sec:pdftex-graphics}).

If you're using \prog{dvips} to get PostScript as an intermediate step
(using \prog{ps2pdf} or Acrobat Distiller to get PDF), you can force
\prog{dvips} to use Type~1 fonts by specifying the \prog{-Ppdf} flag,
as in
\begin{quote}
\begin{verbatim}
unix% dvips -Ppdf foo.dvi -o foo.ps
\end{verbatim}
\end{quote}


\section{Debugging}

One of the trickiest things about using \latex is interpreting
\latex's sometimes cryptic error messages.

In particular, the line numbers that \latex reports are often not the
line numbers where the problem \emph{is}, but the line numbers where
\latex noticed there was a problem.

One useful way of getting a bit more context to help you understand
the problem is to put the line
\begin{quote}
\begin{verbatim}
\setcounter{errorcontextlines}{1000}
\end{verbatim}
\end{quote}
in the preamble of your document, which will provide you with a
(perhaps excessive) amount of context for an error.

The most common errors are
\begin{itemize}
\item Using one of the special characters (see
Section~\ref{sec:special-chars})
\item Leaving off or mismatching a brace or bracket
\item Leaving out or swapping arguments to a command or environment
\end{itemize}

If you've tried everything and you can't find the source of an error
message, try the following procedure:
\begin{enumerate}
\item Create a new file, and copy your preamble into it, followed by
  an empty \env{document} environment
\item Add a word to the \env{document} environment so that \tex will
  have something to typeset
\item Try typesetting the new document---if you have an error, the
  problem is in your preamble
\item If the document typesets, get rid of the single word you'd put
  in the \env{document} environment, copy half of your original
  document's body into the new file, and typeset that
\item If you see your error, then continue halving the document until
  you narrow it down to the problem section
\item If you don't see your error, try the other half
\end{enumerate}

\subsection{The \protect\env{comment} Environment}

If you don't want to go through the process of copying your document,
you can also use the \env{comment} environment, which allows you to
``comment out'' anything between the begin and end environment
commands.  The \env{comment} environment is provided by the
\package{verbatim} package, which also defines new versions of the
\env{verbatim} and \env{verbatim*} environments.

\subsection{The \protect\command{includeonly} Command}

If you've written your document in such a way that each chapter or
other significant subdivision is stored in a separate file and
included in your master document with an \command{include} command,
then you can also use the \command{includeonly} command to limit the
parts of your document that will be typeset to some subset.

\command{includeonly} takes as its argument a comma-separated list of
files that are \command{include}d in your document.  If you place each
item on a separate line, you can comment and comment those lines to
control which files are included in a \latex run.  For example,
\begin{quote}
\begin{verbatim}
\includeonly{
a,
% b,
% c,
d}
\end{verbatim}
\end{quote}
will allow you to typeset the contents of the files \file{a} and
\file{d}, without typesetting the contents of \file{b} and \file{c}.
A nice side-effect is that the auxilliary files for these files are
not changed, so that references to labels in the files not being
typeset will still be expanded in the files that are typeset.  (If the
pagination changes, of course, any \command{pageref} commands will be
incorrect; the same applies if any additional structural commands are
added at the same level as the topmost level in the excluded
files---that is, if you add a new chapter in a file called \file{aa}
between the line that includes \file{a} and \file{b}, the chapter
numbers will be incremented for the typeset files but not for the
files you're not typesetting.

%%% Local Variables: 
%%% mode: latex
%%% TeX-master: "master"
%%% End: 
